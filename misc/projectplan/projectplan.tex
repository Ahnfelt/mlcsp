\documentclass[a4paper,12pt]{article}
%\usepackage{fullpage}
%\usepackage{a4wide}
\usepackage[utf8]{inputenc}
\usepackage[danish, english]{babel}

% For Maple
%\usepackage{maple2e}
%\usepackage{mapleenv}
%\usepackage{mapleplots}
%\usepackage{maplestd2e}
%\usepackage{maplestyle}
%\usepackage{mapletab}
%\usepackage{mapleutil}

% In order to highlight code
\usepackage[pdftex]{color}
\usepackage{listings}

% For graphics support
\usepackage{epsfig}
\usepackage{graphicx}

% Math support
\usepackage{amssymb}
\usepackage{amsmath}
\usepackage{euscript}

% In order to include pdf
\usepackage{pdfpages}

% For algorithm support
\usepackage{algorithm} 
\usepackage{algpseudocode}
%\numberwithin{algorithm}{subsection}
\newcommand{\theHalgorithm}{\arabic{algorithm}}

% For text color
\usepackage{color}

% Pdf section support
\usepackage{hyperref}
\hypersetup{
    bookmarks=true,         % show bookmarks bar?
    unicode=false,          % non-Latin characters in Acrobat's bookmarks
    pdftoolbar=false,       % show Acrobat's toolbar?
    pdfmenubar=false,       % show Acrobat's menu?
    pdffitwindow=true,      % page fit to window when opened
    pdftitle={Algorithms and data structures: homework 1}, % title
    pdfauthor={Ramón Soto Mathiesen - ramon@diku.dk - 15/05/79}, % author
    pdfsubject={Assignment for week 17},   % subject of the document
    pdfnewwindow=true,      % links in new window
    pdfkeywords={keywords}, % list of keywords
    colorlinks=false,       % false: boxed links; true: colored links
    linkcolor=red,          % color of internal links
    citecolor=green,        % color of links to bibliography
    filecolor=magenta,      % color of file links
    urlcolor=cyan           % color of external links
}

% Macros

% Opening
\title{Algorithms and data structures: homework 1}
\author{Joakim Ahnfelt-Rønne - 1986/03/14 - joakim.ahnfelt@gmail.com \and 
        Ramón Salvador Soto Mathiesen - 1979/05/15 - ramon@diku.dk \and
        Andrzej Filinski - andrzej@diku.dk}
\date{Tuesday, 28$^{th}$ April 2009}

\begin{document}

% Line breaks between paragraphs instead of indentation
\parindent=0pt
\parskip=8pt plus 2pt minus 4pt

\maketitle
%\tableofcontents
\newpage

\section*{Background}
When we took the course Extreme Multiprogramming last block, we missed 
a CSP library for a functional programming language. All the presented lan- 
guages was for imperative languages. We missed a library that was based on 
ML, a language we learn and use a lot at the Computer Science Department 
at Copenhagen University.

Our goal is to implement a library for a ML based language. We choose 
OcamML, based on that we could handle ob jects and besides, we would be 
able to crosscompile it, and maybe reuse it on F\#, after question answered 
by F\# main create Don Syme. This way both the scientific community uses 
same tools as the the private sector. 

\section*{Objectives}
\begin{itemize}
 \item Applying the CSP programming model to a functional language (in 
   contrast to imperative lanuages wich has loops, mutable data struc- 
   tures, etc.).
 \item Designing an API using functional programming concepts such as higher 
   order functions and closures.
\end{itemize}
 
\section*{Expected results}
\begin{itemize}
 \item A library for OCaml based on Tony Hoares CSP book and similar to 
   existing libraries such as JCSP, PyCSP and C++CSP, adobted to a 
   functional setting. 
 \item Write an application implementing the 2D ising model using the API.
 \item Concise, understandable documentation of the API.
\end{itemize}

\section*{Expected results}
\begin{itemize}
 \item Socket/process based backend to enable parallelism and distribution.
 \item File format for defining static networks, with functions for loading the 
   network and functions for generating a graphical representation of the 
   network. 
\end{itemize}

\section*{Schedule}
\begin{itemize}
 \item Project start: 2009/05/11
 \item Project stop: 2009/07/19
\end{itemize}

\begin{tabular}{|c|p{11.5cm}|}
    \hline
    Week & Goal\\
    \hline
    1 \& 2 & Implementation of one-to-one, one-to-any and any-to-one channels on
    top of user-level threads, using the POSIX-like mutex and condition
    primitives.\\
    \hline
    3 & Implementation of poison. \\
    \hline
    4 & Implementation of read-guards. \\
    \hline
    5 & Implementation of write-guards.  \\
    \hline
    6 & Something \\
    \hline
    7 & Something \\
    \hline
    8 & Something \\
    \hline
    9 & Something \\
    \hline
\end{tabular}


% ----------------------------------------------------------------------
% Bibliography
% ----------------------------------------------------------------------
\begin{thebibliography}{99}

\bibitem[CSP, Hoare]{hoare}
:\\
Communicating Sequential Processes\\
C. A. R. Hoare\\
Prentice Hall International; (June 24, 2004)\\
ISBN-10: 0-13-153271-5\\
ISBN-13: 978-0-13-153271-7

\end{thebibliography}
% ----------------------------------------------------------------------

\end{document}
