\documentclass[a4paper,12pt]{article}
%\usepackage{fullpage}
%\usepackage{a4wide}
\usepackage[utf8]{inputenc}
\usepackage[danish, english]{babel}

% For Maple
%\usepackage{maple2e}
%\usepackage{mapleenv}
%\usepackage{mapleplots}
%\usepackage{maplestd2e}
%\usepackage{maplestyle}
%\usepackage{mapletab}
%\usepackage{mapleutil}

% In order to highlight code
\usepackage[pdftex]{color}
\usepackage{listings}

% For graphics support
\usepackage{epsfig}
\usepackage{graphicx}

% Math support
\usepackage{amssymb}
\usepackage{amsmath}
\usepackage{euscript}

% In order to include pdf
\usepackage{pdfpages}

% For algorithm support
\usepackage{algorithm} 
\usepackage{algpseudocode}
%\numberwithin{algorithm}{subsection}
\newcommand{\theHalgorithm}{\arabic{algorithm}}

% For text color
\usepackage{color}

% Pdf section support
\usepackage{hyperref}
\hypersetup{
    bookmarks=true,         % show bookmarks bar?
    unicode=false,          % non-Latin characters in Acrobat's bookmarks
    pdftoolbar=false,       % show Acrobat's toolbar?
    pdfmenubar=false,       % show Acrobat's menu?
    pdffitwindow=true,      % page fit to window when opened
    pdftitle={OCamlCSP. A concurrency library for Ocaml}, % title
    pdfauthor={Joakim Ahnfelt-Rønne - 1986/03/14 - joakim.ahnfelt@gmail.com,
        Ramón Salvador Soto Mathiesen - 1979/05/15 - ramon@diku.dk and
        Advisor: Andrzej Filinski - andrzej@diku.dk}, % author
    pdfsubject={},   % subject of the document
    pdfnewwindow=true,      % links in new window
    pdfkeywords={keywords}, % list of keywords
    colorlinks=false,       % false: boxed links; true: colored links
    linkcolor=red,          % color of internal links
    citecolor=green,        % color of links to bibliography
    filecolor=magenta,      % color of file links
    urlcolor=cyan           % color of external links
}

% Macros

% Opening
\title{OCamlCSP. A concurrency library for Ocaml}
\author{Joakim Ahnfelt-Rønne - 1986/03/14 - joakim.ahnfelt@gmail.com \and 
        Ramón Salvador Soto Mathiesen - 1979/05/15 - ramon@diku.dk \and
        Advisor: Andrzej Filinski - andrzej@diku.dk}
\date{Tuesday, 5$^{th}$ May 2009}

\begin{document}

% Line breaks between paragraphs instead of indentation
\parindent=0pt
\parskip=8pt plus 2pt minus 4pt

\maketitle
%\tableofcontents
\newpage

\section*{Background}
When we took the course Extreme Multiprogramming last block, we missed 
a CSP\cite{hoare} library for a functional programming language. All the 
presented libraries were for imperative languages. We missed a library that 
was based on ML, or functional programming in general.

\subsection*{Method and approach}
OCaml comes with several libraries for concurrency.
It provides the basic POSIX-like primitives in the packages Thread,
Mutex and Condition. It provides lightweight threads in the virtual
machine and system threads in both the virtual machine and when
compiled to a native binary. Both thread models offer preemptive
scheduling. Out of the box it also provides an additional library 
with composable channels based on Concurrent ML, called Event. 
% How is this different from CSP?

We're going to base our CSP library on the POSIX-like primitives,
which are almost universally available.

Initially we will build a small core library with a minimal number
of primitives for concurrent computation. We will then expand on this
through multiple iterations, ending each iteration with a slightly
more featured but self contained library with documentation.


\section*{Learning objectives} %Læringsmål.
\begin{itemize}
 \item Designing an API for a CSP-based concurrency library in a functional
   language.
 \item Implementing the concurrency API in a functional language.
 \item Assessing the API and underlying implementation by comparing
   applications built on top of it with similar applications built on
   top of an existing CSP library.
\end{itemize}
The focus is on the API design. Efficiency is not a goal.

\section*{Expected results}
\begin{itemize}
 \item An API for CSP-based concurrent computation in ML.
 \item An implementation of the API in the form of a library for OCaml.
 \item An evaluation of the API (and implementation) comparing
   the library with an existing CSP library such as JCSP, PyCSP or C++CSP,
   with emphasis on design, composability, extendability and usability.
 \item Documentation for the API.
\end{itemize}

\section*{Possible extensions to the project}
\begin{itemize}
 \item True parallelism through system threads (still with shared memory).
 \item Distribution through socket based communication.
 \item File format for defining static networks, similar to the file format 
   approach taken by CSPBuilder for PyCSP.
\end{itemize}

\section*{Schedule}
\begin{itemize}
 \item Project start: 2009/05/11
 \item Project end: 2009/07/31
\end{itemize}

\begin{itemize}
 \item Milestone I (2009/05/31): Simple prototype. Design and implementation of
   a simple API where we can make a simple network, such as the Fibonacci
   example presented in the course XMP 2008/2009. In order to achieve this we
   will at least need:
   \begin{itemize}
     \item processes.
     \item one-to-one channels.
   \end{itemize}
 \item Milestone II (2009/06/30): Final API. In this milestone we should
   have the final version of the API. We will have a code stop in order to be
   able to write the final report. When this milestone is reaches we should be
   able to implement the 2D Ising model or similar.
   \begin{itemize}
     \item channel poisoning.
     \item one-to-any, any-to-one and any-to-any channels.
     \item channel and time guards.
   \end{itemize}
 \item Milestone III (2009/07/31): Report + Final API (refactored). In this
   final milestone we will only make refactoring to the code if we consider so.
   In the report we will also have the final documentation, appendix, for the
   API. The date for this milestone will math the end date for the project.
\end{itemize}

% ----------------------------------------------------------------------
% Bibliography
% ----------------------------------------------------------------------
\begin{thebibliography}{99}

\bibitem[Hoare]{hoare}
:\\
Communicating Sequential Processes\\
C. A. R. Hoare\\
Prentice Hall International; (June 24, 2004)\\
ISBN-10: 0-13-153271-5\\
ISBN-13: 978-0-13-153271-7\\
http://www.usingcsp.com/cspbook.pdf

\end{thebibliography}
% ----------------------------------------------------------------------

\end{document}
