\documentclass[a4paper,12pt]{article}
%\usepackage{fullpage}
%\usepackage{a4wide}
\usepackage[utf8]{inputenc}
\usepackage[danish, english]{babel}

% Nice looking font
%\usepackage{palatino}

% In order to highlight code
\usepackage[pdftex]{color}
\usepackage{listings}

% For graphics support
\usepackage{epsfig}
\usepackage{graphicx}
\usepackage{subfigure}

% Math support
\usepackage{amsmath}

% In order to include pdf
\usepackage{pdfpages}

% Graf support
%\usepackage{tkz-graph}

% Pdf section support
\usepackage{hyperref}
\hypersetup{
    bookmarks=true,         % show bookmarks bar?
    unicode=false,          % non-Latin characters in Acrobat's bookmarks
    pdftoolbar=false,       % show Acrobat's toolbar?
    pdfmenubar=false,       % show Acrobat's menu?
    pdffitwindow=true,      % page fit to window when opened
    pdftitle={OCamlCSP. A concurrency library for Ocaml}, % title
    pdfauthor={Joakim Ahnfelt-Rønne - 1986/03/14 - joakim.ahnfelt@gmail.com,
        Ramón Salvador Soto Mathiesen - 1979/05/15 - ramon@diku.dk and
        Advisor: Andrzej Filinski - andrzej@diku.dk}, % author
    pdfsubject={},   % subject of the document
    pdfnewwindow=true,      % links in new window
    pdfkeywords={keywords}, % list of keywords
    colorlinks=false,       % false: boxed links; true: colored links
    linkcolor=red,          % color of internal links
    citecolor=green,        % color of links to bibliography
    filecolor=magenta,      % color of file links
    urlcolor=cyan           % color of external links
}

% Macros
\newcommand{\missing}[1]{
\begin{tabular}{|p{11cm}|}
\hline
\emph{Missing:} {\scriptsize (things that need to be written or considered)} \\
\hline
#1
\hline
\end{tabular}
}


% Opening
\title{OCamlCSP - a concurrency library for Ocaml}
\author{Joakim Ahnfelt-Rønne - 1986/03/14 - joakim.ahnfelt@gmail.com \and 
        Ramón Salvador Soto Mathiesen - 1979/05/15 - ramon@diku.dk \and
        \\ Advisor: Andrzej Filinski - andrzej@diku.dk}
\date{31$^{st}$ July 2009}

\begin{document}

\maketitle
\newpage
\tableofcontents

% Line breaks between paragraphs instead of indentation
\parindent=0pt
\parskip=8pt plus 2pt minus 4pt

\newpage
\section{Introduction}

\section{Analysis}

\subsection{Interface}

\subsubsection{Design}

\subsubsection{Comparison}

\begin{tabular}{l|l}
CSP & OCamlCSP \\
\hline
$c!e \to P$ & \texttt{write c e; P} \\
\hline
$c?x \to P(x)$ & \texttt{P (read c)} \\
\hline
$(a?x \to P(x)\ |\ b!e \to Q)$ 
&\verb|select [ read_guard a (fun x -> P(x)) ;| \\
&\verb|         write_guard b e (fun _ -> Q) ]|\\
\hline
$P \parallel Q$ 
&\verb|parallel [ (fun () -> P) ;| \\
&\verb|           (fun () -> Q) ]| \\
\end{tabular}

\missing{
- Occam \\
- PyCSP/JCSP/C++CSP \\
- Event \\
}


\section{Implementation}


\subsection{Channels}
A one to one channel is a channel that only allows one procees to read from it and one process 
to write to it. It may be implemented as a state machine as shown in figure \ref{channel-state}.
When the channel is in the NobodyWaiting state and a process tries to read from the channel,
it enters the ReaderWaiting state. When a writer comes along and tries to write to a channel
in this state, it returns to the NobodyWaiting state, transfers the message, and only then
allows the two processes to continue. The WriterWaiting state works in much the same way.

\begin{figure}[h]
\centering
\includegraphics[width=0.6\textwidth]{channel-state.png}
\caption{Simplified channel state (poison and multi-read/write not shown)}
\label{channel-state}
\end{figure}

From any state, the channel might be poisoned and enter the Poisoned state. If it was in the
ReaderWaiting or the WriterWaiting state, the waiting process will wake up and throw a special
exception. Any further attempts to read from or write to the channel will result in this 
exception being thrown again.

For any to any channels that places no restrictions on the number of processes that can read
from or write to it, we simply extend the above to keep a queue of readers (when in the 
ReaderWaiting state) or writers (when in the WriterWaiting state).

The choice of using a queue gives a stronger guarentee than CSP gives, namely that no processes
will be starved. By starvation we mean to say that a process waiting to read or write on the
channel can be blocked indefinately even though an infinite number of messages is transmitted
over the channel. This would happen with a stack if another process is always ready to read or 
write again before the channel was ready to accomodate the starving process. The starving process
would always be put behind the eager process with a stack, but with a queue, the starving process
will always advance in the queue every time a message is transmitted, guarenteeing that it will
eventually get it's turn.

In order to only wake up the waiting process, each process has it's own condition variable that
it waits on. Additionally, readers have a function that takes in a value and performs a side 
effect, while writers have a function that performs a side effect and produces a value. The side
effect for both is to set it's internal reference to the transmitted value (instead of None),
remove it from all the channels it's listening on and wake up the process.

We use a lock that is global to the library, in order to protect the channel and processes from
concurrent mutation. This lock is only ever taken when first trying to read from a channel, or
right after a process has been woken up. We claim that this won't lead to performance degradation,
since OCaml's threads only provides concurrency though time-sharing \cite{ocaml-threads}.

\missing{
- How do you get rid of this lock? (useful for porting). It is hard because you don't know which
channels you will touch before you have chosen a target process to transmit to/from. If you knew
that, you could simply take their locks in some well defined order (thus avoiding deadlock). \\
}

\begin{verbatim}
type 'a channel_state
    = NobodyWaiting 
    | ReaderWaiting of (Condition.t * ('a -> unit)) list
    | WriterWaiting of (Condition.t * (unit -> 'a)) list
    | Poisoned

type ('a, 'b) channel = ('a channel_state) ref
\end{verbatim}

The phantom type 'b is used for the channel permissions. This is enforced solely via the type
system, and is represented with a product of three booleans values: read * write * poison.
Reading, writing and poisoning are initially enabled for a channel, but handles with less 
permissions can be obtained through the functions shown in figure \ref{channel-permissions}. 
Permissions can only be taken away, not gained, which is evident from the types of the 
functions, for example: 

\begin{verbatim}
val read_poison_only : ('a, on * _ * on) channel -> 
                       ('a, on * off * on) channel
\end{verbatim}

Apart from the type, all of these functions are simply identity functions. We provide a
type alias \texttt{'a t = ('a, on * on * on) channel} for people who don't want to write the
longer (often optional) type annotations that the permission types incur.

\begin{figure}[h]
\centering
\begin{tabular}{c|c|c|l}
Read & Write & Poison & Function \\
\hline
0 & 0 & 0 & (useless) \\
0 & 0 & 1 & poison\_only \\
0 & 1 & 0 & write\_only \\
0 & 1 & 1 & write\_poison\_only \\
1 & 0 & 0 & read\_only \\
1 & 0 & 1 & read\_poison\_only \\
1 & 1 & 0 & read\_write\_only \\
1 & 1 & 1 & (default) \\
\end{tabular}
\caption{Channel permissions}
\label{channel-permissions}
\end{figure}


% \section{Channels}
% \missing{
% This section is a short introduction to channels \\
% - Explanation for people who don't know CSP \\
% - Possibly a figure with boxes and lines as a metaphor for processes and channels \\
% }
% 
% \subsection{Communication}
% \missing{
% This section describes the communication primitives \\
% - Synchronization (might be the trickiest part to get right) \\
% - Read, Write, Poision \\
% }
% 
% \begin{figure}[h]
% \centering
% \begin{tabular}{l|l|l}
% CSP & OCamlCSP \\
% \hline
% $c\,?\,x \to P(x)$ & let x = read c in p x \\
% $c\,!\,e \to Q$ & write c e; q () \\
% (no equivalent) & poison c; r () \\
% \end{tabular}
% \caption{Channels in CSP vs. OCamlCSP}
% \label{channel-communication}
% \end{figure}
% 
% \subsection{Alternation}
% \missing{
% This might be it's own section or be merged with communication \\
% - Alternation (and read/write/timeout guards) \\
% - Communication scheduling (prioritized/random select) \\
% - More synchronization \\
% - Actions are automatically performed when chosen (no promises) \\
% - Does random (prioritized (shuffle list)) mean fair? \\
% - Consider having a ``poll'' function that does nothing if none of the alternatives are ready \\
% }
% 
% \begin{figure}[h]
% \centering
% \begin{tabular}{l|l}
% CSP & OCamlCSP \\
% \hline
% & \\
% ($c_1\,?\,x \to P(x) \mid c_2\,?\,y \to Q(y)$) & random [ \\
% & \quad Read (c1, fun x $\to$ p x); \\
% & \quad Read (c2, fun y $\to$ q y) ] \\
% & \\
% ($c_1\,?\,x \to P(x) \mid c_2\,!\,a+b \to Q$) & random [ \\
% & \quad Read (c1, fun x $\to$ p x); \\
% & \quad Write (c2, (fun () $\to$ a + b), fun \_ $\to$ q ()) ] \\
% & \\
% (no equivalent) & random [ \\
% & \quad Read (c1, fun x $\to$ p x); \\
% & \quad Wait (0.100, fun t $\to$ q ()) ] \\
% \end{tabular}
% \caption{Alternation in CSP vs. OCamlCSP}
% \label{channel-alternation}
% \end{figure}
% 
% \subsection{Capabilities}
% \missing{
% This section describes the channels ``end'' capabilities \\
% - Phantom types (with polymorphic variants, informal: channel ends) \\
% - Capabilities can only be taken away, not added \\
% - All (useful) capability combinations are covered \\
% }
% 
% \begin{figure}[h]
% \centering
% \begin{tabular}{c|c|c|l}
% `Read & `Write & `Poison & Function \\
% \hline
% 0 & 0 & 0 & (useless) \\
% 0 & 0 & 1 & poison\_only \\
% 0 & 1 & 0 & write\_only \\
% 0 & 1 & 1 & write\_poison\_only \\
% 1 & 0 & 0 & read\_only \\
% 1 & 0 & 1 & read\_poison\_only \\
% 1 & 1 & 0 & read\_write\_only \\
% 1 & 1 & 1 & (default) \\
% \end{tabular}
% \caption{Channel end capabilities}
% \label{channel-capabilities}
% \end{figure}
% 
% \section{Processes}
% \missing{
% This secion describes the process primitives \\
% - Lambda functions (closures that capture channels) \\
% - Joining \\
% - Captures the return values (functional, not for fork) \\
% - Exception propagation (when all processes are done, not for fork) \\
% - Scope differences from CSP? \\
% - Find out if spawn has an equivalent in CSP. \\
% }
% 
% \begin{figure}[h]
% \centering
% \begin{tabular}{l|l}
% CSP & OCamlCSP \\
% \hline
% & \\
% ($c_1\,?\,x \to skip \parallel c_2\,?\,y \to skip$); $c_3\,!\,x + y$ & let (x, y) = parallel \\
% & \quad (fun () $\to$ read c1) \\
% & \quad (fun () $\to$ read c2) \\
% & in write c3 (x + y) \\
% & \\
% (imperative representation) & let x = ref 0 in \\
% & let y = ref 0 in \\
% & fork [ \\
% & \quad (fun () $\to$ x := read c1); \\
% & \quad (fun () $\to$ y := read c2) ]; \\
% & write c3 (!x + !y) \\
% & \\
% (no equivalent) & spawn 10 (fun i $\to$ \dots) \\
% \end{tabular}
% \caption{Parallel processes in CSP vs. OCamlCSP}
% \label{channel-parallel}
% \end{figure}

\subsection{Alternation}

\missing{
- Occam does not seem to sufficiently describe PRI ALT when two processes with inverse priorities
on the same two channels use it (in the manual). This is why we don't support it. \\
- Consider supporting something a bit like PRI ALT anyway since it's easier to build on top of
than the current randomized select. \\
}

\subsection{Processes}

\section{Test}

\section{Conclusion}

\appendix
\newpage
\section{Appendix: Source code}
\label{appendixSrc}
% ----------------------------------------------------------------------
% At skrive "%" betyder at man kommenterer ligesom som // i C, Java, osv
% ----------------------------------------------------------------------
% set up listing environment with C syntax hightlight
\definecolor{stringcolor}{rgb}{0.50,0.00,0.50}      %
\definecolor{commentcolor}{rgb}{0.00,0.50,0.00}     %
\definecolor{keywordcolor}{rgb}{0.00,0.00,1.00}     %
\definecolor{idcolor}{rgb}{0.00,0.00,0.00}          %
% with \lstset one can predefine parameters for listings
\lstset{language=ML,basicstyle=\ttfamily,keywordstyle=\color{keywordcolor},
        commentstyle={\color{commentcolor}\itshape},
        stringstyle={\color{stringcolor}},
        identifierstyle=\color{idcolor},numbers=left,
        xleftmargin=2em,% framerule=0.8pt,
        stepnumber=1,frame=tlrb,showstringspaces=false,
        firstnumber=1,numberstyle=\ttfamily, breaklines}
% ----------------------------------------------------------------------

\scriptsize
\subsection{API}
\label{appendixAPI}
\subsubsection{Interface (.mli)}
\lstinputlisting{../source/csp.mli}
\subsubsection{Source (.ml)}
\lstinputlisting{../source/csp.ml}
\subsection{CSP Library (Legoland)}
\lstinputlisting{../source/legoland.ml}
\normalsize

\newpage
\section{Appendix: Test}
\scriptsize
\subsection{Milestone I. Fibonacci}
\lstinputlisting{../test/fibCSP.ml}

\subsection{Milestone II. Webproxy}
\lstinputlisting{../source/sockets.ml}
\normalsize

\end{document}
